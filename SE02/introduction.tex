\section{Vorarbeiten}\label{sec:vorarbeiten}
Vor dem Start des ersten Sprints habe ich mich
mit der Anwendung Carlook und dem Framework
Vaadin auseinandergesetzt.
Ich habe im Midterm bereits sehr intensiv mit GitHub gearbeitet und mit Klara kollaboriert.
Der Einstieg war durch die bereits funktionsfähige App
deutlich einfacher und angenehmer als erwartet.

Die Dokumentation wollte ich eigentlich mithilfe einer Action selbst von Markdown zu PDF und HTML rendern, welche
ich im Vorjahr mit Neo Hornberger um einige Features erweitert hatte.
\section{Sprint 1}\label{sec:sprint1}
Das bewerten der user stories in story points fiel anfangs etwas schwer, aber dadurch, dass wir mit 10 Gruppenmitgliedern
deutlich mehr Erfahrung haben konnte Klara die Festlegung recht schleunig koordinieren.
\subsection{Woche 1}\label{sec:woche1}
In Absprache mit Gökhan stand eines der ersten Dev-Issues bereits fest: <u>Alles</u> was mit Autos zusammenhängt muss abgewandelt oder entfernt werden.

Nachdem die Erste View auf unsere App in der LoginView schon fertig war, galt es nun, diese zu verstehen und zu schauen,
wie wir die Registrierung umsetzen.
Dafür habe ich mich mit der Dokumentation von Vaadin Components und Vaadin Flow befasst.
Mit der Wahl von JPA war dort auch klar, dass wir einen Service und ein Repository brauchen würden.


