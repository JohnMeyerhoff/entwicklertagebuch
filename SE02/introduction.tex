\section{Vorarbeiten}\label{sec:vorarbeiten}
Vor dem Start des ersten Sprints habe ich mich
mit der Anwendung Carlook und dem Framework Vaadin auseinandergesetzt.
Ich habe im Midterm bereits sehr intensiv mit GitHub gearbeitet und mit Klara kollaboriert.
Der Einstieg war durch die bereits funktionsfähige App deutlich einfacher und angenehmer als erwartet.

Anfangs habe ich das Carlook Repository in einem Blueprint branch hinzugefügt und die Dokumentation
des Carlook-Programms als ``Blueprint.md`` umbenannt.
Hier ergab sich schon eines der ersten \emph{Issues}, die Entfernung aller auto-bezogenen Aspekte
des Repositorys.

Die Dokumentation wollte ich eigentlich mithilfe einer Action selbst von Markdown zu PDF und HTML rendern, welche
ich im Vorjahr mit Neo Hornberger um einige Features erweitert hatte.


\section{Sprint 1}\label{sec:sprint1}
Das bewerten der user stories in story points fiel anfangs etwas schwer, aber dadurch, dass wir mit 10 Gruppenmitgliedern
deutlich mehr Erfahrung haben konnte Klara die Festlegung recht schleunig koordinieren.
Als erstes Issue hatte ich schon recht früh die Registrierung implementiert.
Dafür hatte ich mich in der Dokumentation von Vaadin in Components und Forms eingelesen.
Der Aufbau des Forms wird wie auch bei den anderen Layouts in Vaadin durch die Reihenfolge bestimmt,
in welcher die Components ( Textfelder und Buttons ) in das Formular eingefügt werden.

\subsection{Woche 1}\label{subsec:woche1}
In Absprache mit Gökhan stand eines der ersten Dev-Issues bereits fest: \emph{Alles}, was mit Autos
zusammenhängt, muss abgewandelt oder entfernt werden.

Nachdem die Erste View auf unsere App in der LoginView schon fertig war, galt es nun, diese zu verstehen und zu schauen,
wie wir die Registrierung umsetzen.
Dafür habe ich mich mit der Dokumentation von Vaadin Components und Vaadin Flow befasst.
Mit der Wahl von JPA war dort auch klar, dass wir einen Service und ein Repository brauchen würden.
Zum Prüfen, ob eine Registrierung erfolgreich war, habe ich eine Backend-View angelegt.
Die Backend-View verwendet das Public-App-Layout und kann somit aktuell von allen Endanwendern
geöffnet werden.
Wir haben uns mit dem Aufbau der Anwendung auseinandergesetzt und ungenutzte blöcken entfernt oder
eingebunden.
Mit \codeline{public static final String WELCOME\_VIEW = "welcome";} wurde die Willkommensseite
beziehungsweise die Route für diese definiert.
Wir hatten uns als Team für den Namen "AcademicFlow" entschieden.
Diesen namen habe ich überall dort eingesetzt, wo zuvor Carlook stand.
Klara hatte zudem ein Logo entworfen, welches mit dem Namen "AcademicFlow" verknüpft wurde.
Auf dem Logo ist ein weißer Hut in einem blauen Kreis mit einem Helleren blauen
Kreis umrandet.
Auch dieses Logo habe ich eingefügt.
Ich habe die CI-Pipeline eingebaut, welche ich schon in SE01 geschrieben hatte.
Diese Pipeline produziert zu den Markdown Dokumenten die PDF-Versionen der Dokumentation.
Diese PDF-Versionen werden in GitHub generiert, für GitLab hat dies leider nicht funktioniert.

\subsection{Woche 2}\label{subsec:woche2}
In dieser Woche habe ich die MVC Architektur in der App durchgesetzt und alle Sources
in die korrekten Verzeichnisse verschoben.
Ich habe im Login-Service in der Methode \codeline{login} ergänzt, dass eine Exception
geworfen wird, wenn die Anmeldung fehlschlägt.
Diese kann dann im Frontend vom Usability-Team in ein Feedback für den User verarbeitet werden.

Als Testmanager habe ich mich mit JUnit befasst und einen Roundtrip-Test für die Registrierung erstellt.
Ich habe auch einen Test für den Login erstellt.
Die Annotationen wie \codeline{@NotNull} und \codeline{@Nullable}
waren zum Testen besonders hilfreich, da direkt aus dem
JavaDoc ersichtlich ist, wann eine Methode null zurückgeben kann.