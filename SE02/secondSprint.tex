

\section{Sprint 2}\label{sec:sprint2}
Nach der Besprechung des ersten Sprints blieben einige Punkte offen, welche es noch ausserhalb der Issues bzw Jira-Tickets zu lösen gab.
Im Sonarqube habe ich viele Fehler in Form von Code-Smells gefunden und untersucht.
Viele Code-Smells ließen sich recht schnell beheben, da die Nachrichten von Sonar sehr verständlich sind.
Ich habe in summe circa 150 Code-Smells entfernt, davon hatte ich ca.\ 40 selber produziert.
Einige Code-Smells konnten auch durch die IDE automatisch entfernt werden, das funktionierte vor allem bei unused Imports.
Die Dokumentation von Vaadin Flow\cite*[]{flowdocumentation}war sehr hilfreich, da die Beispiele ausführlich genug waren,
um die Konzepte zu demonstrieren, ohne zu viel Boilerplate zu verwenden.
GitLab hat eine Einstellung, welche es ermöglicht, merge commits vollständig zu verbieten.
In der Erklärung von Spring und Spring Beans fand ich vor allem dieses Video\cite*[]{springvideo} hilfreich.

\subsection{Woche 1}\label{subsec:s2w1}
In der ersten Woche habe ich die folgenden Aufgaben in erledigt, in

Einzelarbeit:
\begin{itemize}
    \item Einbinden der Font-Dateien in die Anwendung. (Datei und Css-Verlinkung)
    \item Setzen der Fonts für die Elemente (H1 und H2 Allegreya Sans), (Rest Baloo 2)
    \item Erstellen und setzen des Favicons
    \item Entfernen der Code-Smells

\end{itemize}

Gruppenarbeit:

\begin{itemize}
    \item Neue Fehlermeldungen und Prüfung auf E-Mail Duplikat bei Registrierung (mit Tom)
    \item Einarbeitung in Selenium (mit Gökhan und Thomas)
    \item Bereinigung der Design-Fehler (mit Thomas und Klara)
\end{itemize}

\subsection{Woche 2}\label{subsec:s2w2}
In der zweiten Woche habe ich die folgenden Aufgaben in erledigt, in

Einzelarbeit:
\begin{itemize}
    \item Finalisieren des Verticalspacegenerators
    \item JUnit tests für die Logik der Anwendung
    \item Round-Tripping-Test-Pattern verwendet
    \item Stellenanzeigen anzeigen
    \item Stellenanzeigen anzeigen (Detailsicht)

\end{itemize}

Gruppenarbeit:

\begin{itemize}
    \item Finalisieren der Session-Util-Klasse (mit Gökhan)
    \item Auf Stellenanzeigen Bewerben (mit Tom)
    \item Unternehmen bewerten (mit Thomas)
    \item Finalisieren und einbinden der Programmierteile
\end{itemize}
Das Senior Programmer Team bestand aus Sophia, Tom, Gökhan und mir.
Alle anderen sollten einen Programmierteil haben, die fertiggestellten Programmierteile habe ich
in meiner Demo\cite*[]{demo1} dokumentiert.
Der Programmierteil von Klara war visuell sehr ausgereift (Studentenprofil).
Der Programmierteil von Thomas war sehr nützlich für die Entwicklung von Oberflächen, da er uns
eine solide Datenlage generieren ließ.
Jonas hatte als erster mit seinem Programmierteil angefangen.

\subsection{Woche 3}\label{subsec:s2w3}
In der dritten Woche habe ich die folgenden Aufgaben in erledigt, in

Einzelarbeit:
\begin{itemize}
    \item Null-pointer-Exceptions entfernt
    \item Umstieg auf \codeline{Slf4j} (Logging) (mit Gökhan bzw.\ aufgrund seiner Informationen)
    \item Round-Tripping-Tests für alle Entitäten
    \item Anpassungen der Entitäten und deren Constraints
    \item Erweiterungen in den Services für bessere Abfragen
    \item Verbesserungen in Methodennamen und Kapselung (\codeline{private} \textrightarrow \ \codeline{protected})
    \item Programmierfehler korrigiert
\end{itemize}

Gruppenarbeit:

\begin{itemize}
    \item Programmierfehler bzw.\ Bedienfehler im Zusammenhang mit Git behoben (mit Klara)
    \item Finalisieren der Unternehmensbewertung (mit Thomas)
    \item Akzeptanztests für US1 bis US6 (mit Thomas)
\end{itemize}

Einige der schwierigsten Aufgaben waren, die Korrektur von Merge-Commits und das Finden von
fehlerhaften Imports.
Wenn in JPA ein Import von \codeline{jdbc.query} verwendet wurde, so führt dieser in keiner IDE
zu einem Problem, führt jedoch in jeder Ausführung der Anwendung \emph{direkt} zu einem Crash.

\subsection{Woche 4}\label{subsec:s2w4}
In der vierten Woche habe ich die folgenden Aufgaben in erledigt, in

Einzelarbeit:
\begin{itemize}
    \item Programmierfehler behoben
    \item Tests aktualisiert
\end{itemize}

Gruppenarbeit:

\begin{itemize}
    \item Programmierfehler bzw.\ Bedienfehler im Zusammenhang mit Git behoben (mit Klara)
\end{itemize}


