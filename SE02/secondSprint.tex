

\section{Sprint 2}\label{sec:sprint2}
Nach der Besprechung des ersten Sprints blieben einige Punkte offen, welche es noch ausserhalb der Issues bzw Jira-Tickets zu lösen gab.
Im Sonarqube habe ich viele Fehler in Form von Code-Smells gefunden und untersucht.
Viele Code-Smells ließen sich recht schnell beheben, da die Nachrichten von Sonar sehr verständlich sind.
Ich habe in summe circa 150 Code-Smells entfernt, davon hatte ich ca.\ 40 selber produziert.
Einige Code-Smells konnten auch durch die IDE automatisch entfernt werden, das funktionierte vor allem bei unused Imports.
Die Dokumentation von Vaadin Flow\cite*[]{flowdocumentation}war sehr hilfreich, da die Beispiele ausführlich genug waren,
um die Konzepte zu demonstrieren, ohne zu viel Boilerplate zu verwenden.
GitLab hat eine Einstellung, welche es ermöglicht, merge commits vollständig zu verbieten.
In der Erklärung von Spring und Spring Beans fand ich vor allem dieses Video\cite*[]{springvideo} hilfreich.

\subsection{Woche 1}\label{subsec:s2w1}
In der ersten Woche habe ich die folgenden Aufgaben in erledigt, in

Einzelarbeit:
\begin{itemize}
    \item Einbinden der Font-Dateien in die Anwendung. (Datei und Css-Verlinkung)
    \item Setzen der Fonts für die Elemente (H1 und H2 Allegreya Sans), (Rest Baloo 2)
    \item Erstellen und setzen des Favicons
    \item Entfernen der Code-Smells

\end{itemize}

Gruppenarbeit:

\begin{itemize}
    \item Neue Fehlermeldungen und Prüfung auf E-Mail Duplikat bei Registrierung (mit Tom)
    \item Einarbeitung in Selenium (mit Gökhan und Thomas)
    \item Bereinigung der Design-Fehler (mit Thomas und Klara)
\end{itemize}